\documentclass[11pt,a4paper]{scrartcl}
\usepackage[utf8]{inputenc}
\usepackage[utf8]{inputenc}
\usepackage[ngerman]{babel}
\usepackage[T1]{fontenc}
\usepackage{amsmath}
\usepackage{amsfonts}
\usepackage{amssymb}
\usepackage{mathtools}
\usepackage{graphicx}
\usepackage{multirow}
\usepackage{fancyhdr}
\pagestyle{fancy}
\usepackage[a4paper,
			bottom=1.7in,
			left=1.2in,
			right=1.2in,
			top=1.2in,
			headsep = 35pt]
	{geometry}
\usepackage{tikz}% for drawing automata etc
\usetikzlibrary{automata,arrows,chains,shapes.misc,scopes,petri,matrix,patterns}
%\usepackage[x11names]{xcolor}
%\usepackage[parfill]{parskip}
\usepackage{array}
%\usepackage{gslist}
%\usepackage{subfigure}
\usepackage{subcaption}
\usepackage{enumitem}
\usepackage{algpseudocode} %for pseudocode



\setlist[enumerate]{label=\alph*)}
%\setlist[itemize]{label=$\rightarrow$}


 \tikzset{
endbox/.style={pattern=crosshatch,minimum height=.8cm}}

%\partfont{\centering}
\newcommand\tab[1][1cm]{\hspace*{#1}}
\usepackage{framed, color}


\newcommand{\UE}{Übung 1}
\newcommand{\name}{Alexander Halbarth 1129193 alexander.halbarth@student.tuwien.ac.at}
\title{\textbf{Fragenkatalog zum Überprüfungsgespräch Elektrotechnische Grundlagen Übungen für TI 2017 - \UE}}
\author{\name}


\newcommand{\ul}[1]{\underline{#1}} % für unterstreichen von einzelnen Symbolen einfach \ul x - wenn man ein Wort unterstrichen will \ul{wort}
%\newcommand\sy[1]{{\underline{\ifmmode\mathtt{#1}\else\texttt{#1}\fi}}} % ähnlich wie \ul - verwendet aber andere Schriftart
%\newlistt\sys\sy{\hspace{1pt}}{}{}{^} % identisch zu \sy, \sys{wort} unterstreicht aber jeden Buchstaben einzeln und \sy{wort} unterstreicht das ganze Wort
% (sofern das Alphabet nur aus einzelnen Zeichen besteht ist \sys{wort} einfacher


%no line indent
\setlength\parindent{0pt}

\fancyhead[R]{\UE}
\fancyhead[L]{\name}
\fancyfoot[C]{\thepage}

\renewcommand{\footrulewidth}{0pt}
\renewcommand{\headrulewidth}{0.5pt}


\begin{document}
\maketitle

\textbf{Frage 1: Was machst Du gerade im Labor und welchen Sinn hat das?}\\
\textbf{Frage 2: Nenne die Grundgrößn der Elektrotechnik, deren Formelzeichen und Einheit.}\\
\textbf{Frage 2: Elektrische Spannung: Nenne Definition (nicht über das Ohmsche Gesetz!), Formelzeichen und Einheit}\\
\textbf{Frage 2: Elektrischer Strom: Nenne Definition (nicht über das Ohmsche Gesetz!), Formelzeichen und Einheit}\\
\textbf{Frage 2: Elektrischer Widerstand: Nenne Definition (nicht über das Ohmsche Gesetz!), Formelzeichen und Einheit}\\
\textbf{Frage 2: Elektrische Leistung: Nenne Definition, Formelzeichen und Einheit}
\textbf{Frage 2: Wie berechnet man die elektrische Leistung in einem Gleichstromkreis?\\
Berechne die an einem Widerstand entstehende Leistung, wenn durch ihn bei einer Spannung von 2V ein Strom von 3A fließt.}\\
\textbf{Frage 2: Welcher Phasenwinkel besteht zwischen Wechselspannung und Wechselstrom an einem idealen Kondensator?\\
Welcher Phasenwinkel besteht zwischen Wechselspannung und Wechselstrom an einer idealen Induktivität?\\
(Die Vorzeichen brauchen nicht explizit angegeben zu werden, müssen aber verglichen werden).}\\
\textbf{Frage 2: Formuliere das ohmsche Gesetz.\\
Berechne den Widerstand, wenn bei einem Strom von 3A eine Spannung von 3V abfällt.}\\
\textbf{Frage 2: Formuliere das ohmsche Gesetz.\\
Berechne den Strom, wenn an einem Widerstand von 5? eine Spannung von 10V abfällt.}\\
\textbf{Frage 2: Formuliere das ohmsche Gesetz.\\
Berechne die Spannung, wenn durch einen Widerstand von 10? ein Strom von 5A fließt. }\\
\textbf{Frage 2: Formuliere die Kirchhoffschen Regeln.\\
Auf welchem physikalischen Grundprinzip beruhen diese?\\
An einem Spannungsteiler liegen 9V. Am oberen Widerstand liegen 6V an.\\
Berechne die Spannung am unteren Widerstand.}\\
\textbf{Frage 2: Formuliere die Kirchhoffschen Regeln.\\
Auf welchem physikalischen Grundprinzip beruhen diese?\\
In einen Stromknoten mit drei Leitungen fließen aus einer Leitung 2A hinein und aus einer anderen Leitung 3A hinein. Was geschieht in der dritten Leitung?}\\
\textbf{Frage 2: Nenne die Zehnerpotenzen zu den SI - Präfixen Nano, Milli und Mikro.\\
Nenne die SI - Präfixe zu: $10^3, 10^6, 10^9$}\\
\textbf{Frage 3: Benenne die beiden wichtigsten elektrotechnischen Eigenschaften (Größen) eines Widerstandes (konkreter Bauteil) und gib die korrekten Einheiten dazu an.\\
Benenne die wichtigste elektrotechnische Eigenschaft (Größe) eines Kondensators (konkreter Bauteil) und gib die korrekte Einheit dazu an.\\
Benenne die wichtigste elektrotechnische Eigenschaft (Größe) einer Spule (konkreter Bauteil) und gib die korrekte Einheit dazu an.}
\textbf{Frage 3: Wie ist ein Spannungsmessgerät mit einer Schaltung zu verbinden?\\
Wie ist ein Strommessgerät mit einer Schaltung zu verbinden?\\
Welche Maßnahme ist logischerweise nach Abschluss einer Strommessung notwendig?}\\
\textbf{Frage 3: Was ist eine Kennlinie?\\
Skizziere die Kennlinie eines ohmschen Widerstandes.\\
Skizziere die Kennlinie einer Diode.}\\
\textbf{Frage 3: Nenne die drei Kenngrößn sinusförmiger gleichspannungsfreier Wechselspannung!\\
Benenne auch die korrekten Einheiten dazu!}\\
\textbf{Frage 4: Es gibt drei Arten der Messung von Widerstandswerten. Benenne und beschreibe sie (Schaltskizze).}\\
\textbf{Frage 4: Ein Spannungsteiler 1k? zu 3k? liegt an einer Spannung von 8V. Skizziere die Schaltung und berechne die Spannungen an den beiden Widerständen.}\\
\textbf{Frage 4: Ein Spannungsteiler 1k? zu 2k? parallel 2k? liegt an einer Spannung von 8V. Skizziere die Schaltung und berechne die beiden Teilspannungen.}\\
\textbf{Frage 4: Berechne die unbekannte Spannung in folgender Schaltung: Welches Prinzip verwendest Du dabei? (2 Antworten zulässig.)}\\
\textbf{Frage 4: Berechne den unbekannten Strom in folgender Schaltung: Welches Prinzip verwendest Du dabei? (2 Antworten zulässig.)}\\
\textbf{Frage 5: Was ist ein Oszilloskop?}\\
\textbf{Frage 5: Benenne die beiden wichtigsten Bedienungselemente des Oszilloskops für die Y - Achse.\\
Beschreibe die Methode der Amplitudenmessung mit dem Oszilloskop gemäß Bild rechts.}\\
\textbf{Frage 5: Benenne die beiden wichtigsten Bedienungselemente des Oszilloskops für die X - Achse.\\
Beschreibe die Methode der Periodenzeitmessung mit dem Oszilloskop gemäß Bild rechts.}\\
\textbf{Frage 5: Beschreibe Zweck, elementare Bedienungselemente und elementare Bedienung der Trigger - Einheit des Oszilloskops.}\\
\textbf{Frage 5: Wozu dient ein Funktionsgenerator?\\
Nenne drei wichtige Einstellungen eines Funktionsgenerators.}\\
\textbf{Frage 5: Benenne drei wichtige vom Funktionsgenerator gelieferte Signalformen auf Deutsch und Englisch und skizziere diese.}\\
\textbf{Frage 5: Was ist der Unterschied zwischen Y - T und X - Y Betrieb eines Oszilloskops?}\\
\textbf{Frage 5: Warum muss in dieser Schaltung Kanal Y1 des Oszilloskops invertiert geschalten werden?\\
Warum darf bei solchen Aufbauten die Masse des Funktionsgenerators nicht mit Erde verbunden sein?}\\
\textbf{Frage 5: Die Werte des Widerstandes, des Kondensators und der Frequenz werden so gewählt, dass die Amplituden der Spannungen X und Y1 gleich sind.\\
Die Wechselspannung ist sinusförmig.\\
Das Oszilloskop ist in Betriebsart X-Y geschalten.\\
Welche Kurve zeigt das Oszilloskop?\\
Wie kann man diese Kurve mathematisch erklären?}\\
\end{document}
