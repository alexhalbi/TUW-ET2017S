\documentclass[11pt,a4paper]{scrartcl}
\usepackage[utf8]{inputenc}
\usepackage[utf8]{inputenc}
\usepackage[ngerman]{babel}
\usepackage[T1]{fontenc}
\usepackage{amsmath}
\usepackage{amsfonts}
\usepackage{amssymb}
\usepackage{mathtools}
\usepackage{graphicx}
\usepackage{multirow}
\usepackage{fancyhdr}
\usepackage{float}
\pagestyle{fancy}
\usepackage[a4paper,
			bottom=1.7in,
			left=1.2in,
			right=1.2in,
			top=1.2in,
			headsep = 35pt]
	{geometry}
\usepackage{tikz}% for drawing automata etc
\usetikzlibrary{automata,arrows,chains,shapes.misc,scopes,petri,matrix,patterns}
%\usepackage[x11names]{xcolor}
%\usepackage[parfill]{parskip}
\usepackage{array}
%\usepackage{gslist}
%\usepackage{subfigure}
\usepackage{subcaption}
\usepackage{enumitem}
\usepackage{algpseudocode} %for pseudocode
\usepackage[siunitx,european,straightlabels,straightvoltages]{circuitikz} %draw Electrical Circuits
\usepackage[miktex]{gnuplottex}

\input{../mymeters.tex}

\author{Alexander Halbarth}

\setlist[enumerate]{label=\alph*)}
%\setlist[itemize]{label=$\rightarrow$}


 \tikzset{
endbox/.style={pattern=crosshatch,minimum height=.8cm}}

%\partfont{\centering}
\newcommand\tab[1][1cm]{\hspace*{#1}}
\usepackage{framed, color}


\newcommand{\UE}{Übung 4}
\newcommand{\name}{Fragenkatalog zum Überprüfungsgespräch Elektrotechnische Grundlagen}
\title{\textbf{Fragenkatalog zum Überprüfungsgespräch Elektrotechnische Grundlagen Übungen für TI 2017 - \UE}}


\newcommand{\ul}[1]{\underline{#1}} % für unterstreichen von einzelnen Symbolen einfach \ul x - wenn man ein Wort unterstrichen will \ul{wort}
%\newcommand\sy[1]{{\underline{\ifmmode\mathtt{#1}\else\texttt{#1}\fi}}} % ähnlich wie \ul - verwendet aber andere Schriftart
%\newlistt\sys\sy{\hspace{1pt}}{}{}{^} % identisch zu \sy, \sys{wort} unterstreicht aber jeden Buchstaben einzeln und \sy{wort} unterstreicht das ganze Wort
% (sofern das Alphabet nur aus einzelnen Zeichen besteht ist \sys{wort} einfacher


%no line indent
\setlength\parindent{0pt}

\fancyhead[R]{\UE}
\fancyhead[L]{\name}
\fancyfoot[C]{\thepage}

\renewcommand{\footrulewidth}{0pt}
\renewcommand{\headrulewidth}{0.5pt}


\begin{document}
\maketitle
\textbf{Frage 1: Was machst Du gerade im Labor und welchen Sinn hat das?}\\
\textbf{Frage 2: Nenne die Grundgrößen der Elektrotechnik, deren Formelzeichen und Einheit.}\\
Spannung $U[$Volt $V]$, Strom $I[$Ampere $A]$, Widerstand $R[$Ohm $\Omega]$,Leistung $P[$Watt $W]$\\
\textbf{Frage 2: Elektrische Spannung: Nenne Definition (nicht über das Ohmsche Gesetz!), Formelzeichen und Einheit}\\
Die potentielle Energie(=Arbeit) die durch eine Ladungstrennung gespeichert wurde.\\
\textbf{Frage 2: Elektrischer Strom: Nenne Definition (nicht über das Ohmsche Gesetz!), Formelzeichen und Einheit}\\
Ladung eines Elektrons $Q_e=1,6 \cdot 10^{-19}C=1,6 \cdot 10^{-19}A\cdot s$\\
$\frac{6*10^{18}e^-}{1s}=\frac{As}{1s} \rightarrow 6 \cdot 10^{18}$ Elektronen fließen durch einen Leiter pro Sekunde bei $1A$\\
\textbf{Frage 2: Elektrischer Widerstand: Nenne Definition (nicht über das Ohmsche Gesetz!), Formelzeichen und Einheit}\\
Ist eine Materialeigenschaft\\
Beispiel Kupfer $17m\Omega \cdot mm^2/m$\\
\textbf{Frage 2: Elektrische Leistung: Nenne Definition, Formelzeichen und Einheit}\\
Die in einer Zeitspannung umgesetzte elektrische Energie bezogen auf die Zeitspanne.\\
\textbf{Frage 2: Wie berechnet man die elektrische Leistung in einem Gleichstromkreis?}\\
$U=R \cdot I \rightarrow P=U \cdot I$\\
\textbf{Berechne die an einem Widerstand entstehende Leistung, wenn durch ihn bei einer Spannung von $2V$ ein Strom von $3A$ fließt.}\\
$6W$\\
\textbf{Frage 2: Welcher Phasenwinkel besteht zwischen Wechselspannung und Wechselstrom an einem idealen Kondensator?}\\
Die Spannung folgt dem Strom um $90^\circ=\pi/2$ nach. (eigentlich $-90^\circ$)\\
\textbf{Welcher Phasenwinkel besteht zwischen Wechselspannung und Wechselstrom an einer idealen Induktivität?\\
(Die Vorzeichen brauchen nicht explizit angegeben zu werden, müssen aber verglichen werden).}\\
Der Strom folgt der Spannung nach um $90^\circ=\pi/2$ nach. (eigentlich $+90^\circ$)\\
\textit{Gehen in entgegengesetzte Richtungen!}\\
\textbf{Frage 2: Formuliere das ohmsche Gesetz.}\\
$U=R*I$\\
\textbf{Berechne den Widerstand, wenn bei einem Strom von $3A$ eine Spannung von $3V$ abfällt.}\\
\textbf{Frage 2: Berechne den Strom, wenn an einem Widerstand von $5\Omega$ eine Spannung von $10V$ abfällt.}\\
\textbf{Frage 2: Berechne die Spannung, wenn durch einen Widerstand von $10\Omega$ ein Strom von $5A$ fließt. }\\
\textbf{Frage 2: Formuliere die Kirchhoffschen Regeln.}\\
\textbf{Auf welchem physikalischen Grundprinzip beruhen diese?}\\
\textit{Energieerhaltung}\\
Maschenregel: Summe aller Spannungen muss 0 ergeben.\\
Knotenregel: Summe aller Ströme muss 0 ergeben.\\
\textbf{An einem Spannungsteiler liegen $9V$. Am oberen Widerstand liegen $6V$ an.}\\
\textbf{Berechne die Spannung am unteren Widerstand.}\\
\textbf{Frage 2: In einen Stromknoten mit drei Leitungen fließen aus einer Leitung $2A$ hinein und aus einer anderen Leitung $3A$ hinein. Was geschieht in der dritten Leitung?}\\
\textbf{Frage 2: Nenne die Zehnerpotenzen zu den SI - Präfixen Nano, Milli und Mikro. Nenne die SI - Präfixe zu: $10^3, 10^6, 10^9$}\\
\begin{tabular}{|l|l|l|}
\hline
\textbf{Präfix} & \textbf{Zeichen} & \textbf{Faktor}\\
\hline\hline
Piko   & p  & $10^{-12}$\\
\hline
Nano   & n  & $10^{-9}$\\
\hline
Mikro  & $\mu$ & $10^{-6}$\\
\hline
Milli  & m  & $10^{-3}$\\
\hline
Zenti  & c  & $10^{-2}$\\
\hline
Dezi   & d  & $10^{-1}$\\
\hline\hline
Deka   & da & $10^{1}$\\
\hline
Hekto  & h  & $10^{2}$\\
\hline
Kilo   & k  & $10^{3}$\\
\hline
Mega   & M  & $10^{6}$\\
\hline
Giga   & G  & $10^{9}$\\
\hline
Tera   & T  & $10^{12}$\\
\hline
\end{tabular}\\
\newpage
\textbf{Frage 3: Ein Sensor hat einen Ausgangswiderstand von $10k\Omega$. Das Signal wird über ein $100m$ langes Koaxialkabel mit einer Kapazität von $100pF/m$ übertragen. Der Sensor liefert ein symmetrisches Rechtecksignal. Schätze die höchste Frequenz dieses Signals ab, das über diese Strecke mit
akzeptabler Qualität übertragen werden kann.}\\

\\

\textbf{Frage 3: Ein Sensor hat einen Ausgangswiderstand von $10k\Omega$. Das Signal wird über ein $100m$ langes Koaxialkabel mit einer Kapazität von $100pF/m$ übertragen. Das Ausgangssignal des Sensors ist ein Sprung von $0.0V$ auf $+3.16V$. Nach welcher Zeit erkennt der Empfänger den Sprung, wenn er bei einer
Spannung von etwa $+2.00V$ schaltet.}

\\

\textbf{Frage 3: Ein Sensor hat einen Ausgangswiderstand von $10k\Omega$. Das Signal wird über ein $100m$ langes Koaxialkabel mit einer Kapazität von $100pF/m$ übertragen. Der Sensor liefert ein Sinussignal von etwa 3.3Vpp. Bei welcher Frequenz kommen am Empfänger noch etwa 2.33Vpp an?}

\\

\textbf{Welche Phasenverschiebung hat das Signal auf seinem Weg durch das Kabel
bei dieser Frequenz?}

\\

\textbf{Frage 3: Ein Sensor hat einen Ausgangswiderstand von $10k\Omega$. Das Signal wird über ein $100m$ langes Koaxialkabel mit einer Kapazität von $100pF/m$ übertragen. Der Sensor liefert ein unsymmetrisches Rechtecksignal: $+10V$ für $100\mu s$, dann $-10V$ für $900\mu s$. Das Signal am Empfänger muss $99\%$ des Sensorsignals sein. Kann dieses Signal über dieses Kabel korrekt übertragen werden?}

\\

\textbf{Frage 3: Ein Taster liefert folgendes Signal:}\\
\includegraphics{Taster.png}\\
\\

\textbf{Mit welchem Filter kannst Du dieses Prellen aus dem Signal entfernen?}

\\

\textbf{Mit welcher Schaltung kannst Du danach die Flankensteilheit wieder erhöhen?}

\\

\textbf{Frage 3: Erkläre das Verhalten dieser Schaltung:}\\
\includegraphics{Schaltung_1.png}\\
\\

\textbf{Frage 3: Skizziere die Übertragungsfunktion dieser Schaltung:\\
(Doppelt logarithmischer Maßstab der Achsen)}\\
\includegraphics{Schaltung_2.png}\\
\\

\textbf{Frage 4: Vergleiche das Bode‐Diagramm mit einem Spektrum. Was sind die Gemeinsamkeiten?}

\\

\textbf{Was wird womit beschrieben?}

\\

\textbf{Frage 4: Elektrische Spannung kann unter anderem in $V_{PP}$, $V_{RMS}$ und $dBV_{RMS}$ angegeben werden. Beschreibe die Anwendungszwecke und Unterschiede zwischen diesen Arten der Quantifizierung.}\\

\\

\textbf{Eine sinusförmige Spannung hat einen Maximalwert von $+5V$ und einen
Minimalwert von $-5V$. Berechne die Spannung in $V_{PP}$, $V_{RMS}$ und $dBV_{RMS}$}\\

\textbf{Frage 4: Elektrische Spannung kann unter anderem in $V_{SS}$, $V_{EFF}$ und $dBV_{EFF}$ angegeben werden. Beschreibe die Anwendungszwecke und Unterschiede zwischen diesen Arten der Quantifizierung.}\\

\\

\textbf{Eine sinusförmige Spannung hat einen Maximalwert von $+5V$ und einen
Minimalwert von $-5V$. Berechne die Spannung in $V_{SS}$, $V_{EFF}$ und $dBV_{EFF}$}\\

\textbf{Frage 4: Gegeben ist eine konstante Kosinusspannung der Amplitude A und der Kreisfrequenz $\omega_S : U(t) = A cos (\omega_St)$\\Stelle das kontinuierliche Fourier ‐ Integral dazu auf. }\\


\textbf{Existiert dieses Integral überhaupt? Begründe deine Entscheidung!}\\


\textbf{Frage 4: Wozu braucht man in der praktischen Fourier-Analyse Fensterfunktionen? Nach welchen Kriterien wählst Du die Fensterfunktion für Deine Messaufgabe?}\\


\textbf{Frage 4: Was ist der Unterschied zwischen Fourier-Reihe und Fourier-Integral? }\\

\textbf{Was ist der Unterschied zwischen kontinuierlicher-, diskreter- und schneller Fourier-Transformation?}

\\

\textbf{Frage 5: Du misst eine unbekannte periodische Eingangsspannung mit einem digitalen Oszilloskop. Bei einer Horizontalablenkung $1ms/DIV$ siehst Du eine volle Periode eines schönen Sinussignals. Das Bild wandert langsam von links nach rechts. Welche Schlussfolgerung ziehst Du aus diesem Bild?}
	
\\

\textbf{Was kannst Du machen, um fatale Fehlinterpretationen zu vermeiden?}
	
\textbf{Frage 5: Du sollst ein unsymmetrisches $1kHz$ Rechtecksignal $+10V$ für $100\mu s$, dann $-10V$ für $900\mu s$ effektiv digitalisieren. Wähle eine geeignete Abtastrate und begründe Deine Entscheidung.}

\\

\textbf{Frage 5: Was ist das Gibbs'sche Phänomen? Was kann man praktisch gegen die Auswirkungen dieses Phänomens tun?}

\\

\textbf{Frage 5: Ein Sensor liefert ein Signal, bestehend aus den beiden Sinusfunktionen $sin (10Hz \cdot t)$ und $sin(11Hz \cdot t)$. Dein erstes Übertragungssystem ist perfekt linear mit der Übertragungsfunktion
$U_{a} = 3 \cdot U_{in}$. Dein zweites Übertragungssystem ist nichtlinear mit der Übertragungsfunktion $U_a =  U_{in}^2$. \\Hinweise: $sin^2(x)=\frac{1}{2}(1-cos(2x)$\\$sin(x)\cdot sin(y) = \frac{1}{2}(cos(x-y)-cos(x+y))$\\ Skizziere die beiden aus den Übertragungssystemen resultierenden Spektren und
interpretiere dieses Ergebnis! (Hinweis: Eine komplette Berechnung ist nicht erforderlich!).}
\end{document}
