\documentclass[11pt,a4paper]{scrartcl}
\usepackage[utf8]{inputenc}
\usepackage[utf8]{inputenc}
\usepackage[ngerman]{babel}
\usepackage[T1]{fontenc}
\usepackage{amsmath}
\usepackage{amsfonts}
\usepackage{amssymb}
\usepackage{mathtools}
\usepackage{graphicx}
\usepackage{multirow}
\usepackage{fancyhdr}
\pagestyle{fancy}
\usepackage[a4paper,
			bottom=1.7in,
			left=1.2in,
			right=1.2in,
			top=1.2in,
			headsep = 35pt]
	{geometry}
\usepackage{tikz}% for drawing automata etc
\usetikzlibrary{automata,arrows,chains,shapes.misc,scopes,petri,matrix,patterns}
%\usepackage[x11names]{xcolor}
%\usepackage[parfill]{parskip}
\usepackage{array}
%\usepackage{gslist}
%\usepackage{subfigure}
\usepackage{subcaption}
\usepackage{enumitem}
\usepackage{algpseudocode} %for pseudocode

\author{Alexander Halbarth}

\setlist[enumerate]{label=\alph*)}
%\setlist[itemize]{label=$\rightarrow$}


 \tikzset{
endbox/.style={pattern=crosshatch,minimum height=.8cm}}

%\partfont{\centering}
\newcommand\tab[1][1cm]{\hspace*{#1}}
\usepackage{framed, color}


\newcommand{\UE}{Übung 1}
\newcommand{\name}{Fragenkatalog zum Überprüfungsgespräch Elektrotechnische Grundlagen}
\title{\textbf{Fragenkatalog zum Überprüfungsgespräch Elektrotechnische Grundlagen Übungen für TI 2017 - \UE}}


\newcommand{\ul}[1]{\underline{#1}} % für unterstreichen von einzelnen Symbolen einfach \ul x - wenn man ein Wort unterstrichen will \ul{wort}
%\newcommand\sy[1]{{\underline{\ifmmode\mathtt{#1}\else\texttt{#1}\fi}}} % ähnlich wie \ul - verwendet aber andere Schriftart
%\newlistt\sys\sy{\hspace{1pt}}{}{}{^} % identisch zu \sy, \sys{wort} unterstreicht aber jeden Buchstaben einzeln und \sy{wort} unterstreicht das ganze Wort
% (sofern das Alphabet nur aus einzelnen Zeichen besteht ist \sys{wort} einfacher


%no line indent
\setlength\parindent{0pt}

\fancyhead[R]{\UE}
\fancyhead[L]{\name}
\fancyfoot[C]{\thepage}

\renewcommand{\footrulewidth}{0pt}
\renewcommand{\headrulewidth}{0.5pt}


\begin{document}
\maketitle

\textbf{Frage 1: Was machst Du gerade im Labor und welchen Sinn hat das?}\\
\textbf{Frage 2: Nenne die Grundgrößen der Elektrotechnik, deren Formelzeichen und Einheit.}\\
Spannung $U[$Volt $V]$, Strom $I[$Ampere $A]$, Widerstand $R[$Ohm $\Omega]$,Leistung $P[$Watt $W]$\\
\textbf{Frage 2: Elektrische Spannung: Nenne Definition (nicht über das Ohmsche Gesetz!), Formelzeichen und Einheit}\\
Die potentielle Energie(=Arbeit) die durch eine Ladungstrennung gespeichert wurde.\\
\textbf{Frage 2: Elektrischer Strom: Nenne Definition (nicht über das Ohmsche Gesetz!), Formelzeichen und Einheit}\\
Ladung eines Elektrons $Q_e=1,6 \cdot 10^{-19}C=1,6 \cdot 10^{-19}A\cdot s$\\
$\frac{6*10^{18}e^-}{1s}=\frac{As}{1s} \rightarrow 6*10^{18}$ Elektronen fließen durch einen Leiter pro Sekunde bei $1A$\\
\textbf{Frage 2: Elektrischer Widerstand: Nenne Definition (nicht über das Ohmsche Gesetz!), Formelzeichen und Einheit}\\
Ist eine Materialeigenschaft\\
Beispiel Kupfer $17m\Omega \cdot mm^2/m$\\
\textbf{Frage 2: Elektrische Leistung: Nenne Definition, Formelzeichen und Einheit}\\
Die in einer Zeitspannung umgesetzte elektrische Energie bezogen auf die Zeitspanne.\\
\textbf{Frage 2: Wie berechnet man die elektrische Leistung in einem Gleichstromkreis?}\\
$U=R \cdot I \rightarrow P=U \cdot I$\\
\textbf{Berechne die an einem Widerstand entstehende Leistung, wenn durch ihn bei einer Spannung von $2V$ ein Strom von $3A$ fließt.}\\
$6W$\\
\textbf{Frage 2: Welcher Phasenwinkel besteht zwischen Wechselspannung und Wechselstrom an einem idealen Kondensator?}\\
Die Spannung folgt dem Strom um $90^\circ=\pi/2$ nach. (eigentlich $-90^\circ$)\\
\textbf{Welcher Phasenwinkel besteht zwischen Wechselspannung und Wechselstrom an einer idealen Induktivität?\\
(Die Vorzeichen brauchen nicht explizit angegeben zu werden, müssen aber verglichen werden).}\\
Der Strom folgt der Spannung nach um $90^\circ=\pi/2$ nach. (eigentlich $+90^\circ$)\\
\textit{Gehen in entgegengesetzte Richtungen!}\\
\textbf{Frage 2: Formuliere das ohmsche Gesetz.}\\
$U=R*I$\\
\textbf{Berechne den Widerstand, wenn bei einem Strom von $3A$ eine Spannung von $3V$ abfällt.}\\
\textbf{Frage 2: Berechne den Strom, wenn an einem Widerstand von $5\Omega$ eine Spannung von $10V$ abfällt.}\\
\textbf{Frage 2: Berechne die Spannung, wenn durch einen Widerstand von $10\Omega$ ein Strom von $5A$ fließt. }\\
\textbf{Frage 2: Formuliere die Kirchhoffschen Regeln.}\\
\textbf{Auf welchem physikalischen Grundprinzip beruhen diese?}\\
\textit{Energieerhaltung}\\
Maschenregel: Summe aller Spannungen muss 0 ergeben.\\
Knotenregel: Summe aller Ströme muss 0 ergeben.\\
\textbf{An einem Spannungsteiler liegen $9V$. Am oberen Widerstand liegen $6V$ an.}\\
\textbf{Berechne die Spannung am unteren Widerstand.}\\
\textbf{Frage 2: In einen Stromknoten mit drei Leitungen fließen aus einer Leitung $2A$ hinein und aus einer anderen Leitung $3A$ hinein. Was geschieht in der dritten Leitung?}\\
\textbf{Frage 2: Nenne die Zehnerpotenzen zu den SI - Präfixen Nano, Milli und Mikro. Nenne die SI - Präfixe zu: $10^3, 10^6, 10^9$}\\
\begin{tabular}{|l|l|l|}
\hline
\textbf{Präfix} & \textbf{Zeichen} & \textbf{Faktor}\\
\hline\hline
Piko   & p  & $10^{-12}$\\
\hline
Nano   & n  & $10^{-9}$\\
\hline
Mikro  & $\mu$ & $10^{-6}$\\
\hline
Milli  & m  & $10^{-3}$\\
\hline
Zenti  & c  & $10^{-2}$\\
\hline
Dezi   & d  & $10^{-1}$\\
\hline\hline
Deka   & da & $10^{1}$\\
\hline
Hekto  & h  & $10^{2}$\\
\hline
Kilo   & k  & $10^{3}$\\
\hline
Mega   & M  & $10^{6}$\\
\hline
Giga   & G  & $10^{9}$\\
\hline
Tera   & T  & $10^{12}$\\
\hline
\end{tabular}\\
\textbf{Frage 3: Skizziere Kurvenform und Spektrum eines Sinus / symmetrischen Rechteck / Impuls - Signals.\\
Hinweis: Du legst fest, was genau Du unter "`Impuls"' verstehst. Das Spektrum muss zur angegebenen Kurvenform passen!}\\
\\
\textbf{Frage 3: Skizziere den Verlauf des Quotienten U / I eines ohmschen Widerstandes}\\

\textbf{Begründe Deine Skizzen durch die Angabe der Berechnungsformeln.}\\

\textbf{/ eines idealen Kondensators} \\

\textbf{/ einer idealen Spule als Funktion der Frequenz.}\\


\textbf{Frage 3: Erkläre die Bedeutung von $dB$.}\\

\textbf{Welches Spannungsverhältnis wird durch $20dB$ ausgedrückt?}\\

\textbf{Frage 3: In einen Stromknoten mit drei Leitungen fließen aus einer Leitung $2A$ sinusförmiger $50Hz$ Wechselstrom hinein und aus einer anderen Leitung $3A$ sinusförmiger $50Hz$ Wechselstrom hinein. \\
Kannst Du anhand dieser Angaben berechnen, was in der dritten Leitung geschieht? Begründe Deine Entscheidung!}\\
Ein Wechselstrom von $5A$ fließt hinaus. Da die Knotenregel auch im Wechselstrom gilt.\\
\textbf{Frage 4: Skizziere die Schaltung eines RC – Hochpassfilters. Gib seine Grenzfrequenz an.}\\
\\
\textbf{Frage 4: Skizziere die Schaltung eines RC – Tiefpassfilters. Gib seine Grenzfrequenz an.}
\\
\textbf{Frage 4: Skizziere die Schaltung eines RL – Hochpassfilters. Gib seine Grenzfrequenz an.}\\
\\
\textbf{Frage 4: Skizziere die Schaltung eines RL - Tiefpassfilters. Gib seine Grenzfrequenz an.}\\
\\
\textbf{Frage 4: Skizziere die Schaltung eines RLC - Bandpassfilters. (2 Antworten zulässig) Gib seine Mittenfrequenz an.}\\
\\
\textbf{Frage 4: Skizziere das Bode-Diagramm eines RC - Hochpassfilters. Achte auf die korrekte Achsenteilung!}\\
\\
\textbf{Frage 4: Skizziere die Sprungantwort eines Tiefpassfilters 1. Ordnung.}\\
\\
\textbf{Skizziere die Sprungantwort eines Hochpassfilters 1. Ordnung.}\\
\\
\textbf{Begründe Deine Darstellungen.}\\
\\
\textbf{Frage 4: Durch welche beiden Messungen lassen sich Filter besonders effektiv charakterisieren?}
\\
\textbf{Wie gehst Du dabei praktisch vor?}\\
\\
\textbf{Frage 4: Wie verhält sich die Ausgangsspannung eines Tiefpassfilters bei sinusförmigen Eingangsspannungen mit Frequenzen weit unter der Grenzfrequenz?}\\
\\
\textbf{Wie verhält sich die Ausgangsspannung eines Tiefpassfilters bei rechteckförmigen Eingangsspannungen mit Periodenzeiten kleiner als die Zeitkonstante?}\\
\\
\textbf{Frage 4: Wie verläuft die Übertragungsfunktion eines Tiefpassfilters bei Frequenzen deutlich höher als die Grenzfrequenz?}\\
\\
\textbf{Durch welche elektrotechnische Größe wird dieser Verlauf quantifiziert?}\\
\\
\textbf{Frage 4: Wie verläuft die Übertragungsfunktion eines Hochpassfilters bei Frequenzen deutlich niedriger als die Grenzfrequenz?}\\
\\
\textbf{Durch welche elektrotechnische Größe wird dieser Verlauf quantifiziert?}\\
\\
\textbf{Frage 5: Welchen Vorteil haben Tiefpassfilter 2. Ordnung gegenüber Tiefpassfiltern 1. Ordnung?}\\
\\
\textbf{Skizziere die Übertragungsfunktionen im Frequenzbereich.}\\
\\
\textbf{Frage 5: Dieses Diagramm zeigt zwei Sprungantworten eines dynamischen Systems 2. Ordnung, beispielsweise eines LCR - Bandpassfilters.\\
Was sagen diese beiden Sprungantworten über die Dämpfung aus? \\
Wie ist das möglich, dass eine negative Spannung auftritt, obwohl nur passive Bauelemente verwendet werden?}\\
\includegraphics[width=4cm,keepaspectratio]{sprungantwort.jpg}\\
\\
\end{document}
